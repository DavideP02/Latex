% pacchetti fondamentali per qualsiasi documento
\usepackage[T1]{fontenc}
\usepackage[utf8]{inputenc}
\usepackage[italian]{babel}
\usepackage[babel]{csquotes}
\usepackage[style=numeric]{biblatex}
\usepackage{microtype}

\usepackage{centernot}
\usepackage[outline]{contour}
\contourlength{3pt}

\usepackage{fancyhdr}
\usepackage{tikz}
\usetikzlibrary{matrix, cd, patterns, calc, decorations.pathreplacing}
\usepackage{mathrsfs} %per geogebra
\tikzset
{
  %surface/.style={fill=black!10, shading=ball,fill opacity=0.4},
  plane/.style={black,pattern=north east lines},
  curve/.style={black,line width=0.5mm},
}


\usepackage{graphicx} % inserire immagini
\usepackage{multicol} % due colonne
\usepackage{ulem} % sottolineare
\usepackage{lipsum} % lorem ipsum
\usepackage{xcolor} % colori in latex

\usepackage[hang, perpage, symbol*]{footmisc} %per le note a pié pagina
\footnotemargin=0.8em
\DefineFNsymbolsTM{myfnsymbols}{% def. from footmisc.sty "bringhurst" symbols
  \textdagger    \dagger
  \textdaggerdbl \ddagger
  \textsection   \mathsection
  \textbardbl    \|%
  \textparagraph \mathparagraph
  \textdagger\textdagger \dagger\dagger
  \textdaggerdbl\textdaggerdbl \ddagger\ddagger
  \textsection\textsection \mathsection\mathsection
  \textparagraph\textparagraph \mathparagraph\mathparagraph
}%
\setfnsymbol{myfnsymbols}
%comandi per footnotemark consecutivi
	\newcommand{\footnotemarke}{%
	\begin{NoHyper}
		\footnotemark
	\end{NoHyper}}
	\newcommand{\footnotemarkk}[1]{%
		\hyperref[#1]{\footnotemarke}}
	\newcommand{\consecfoottext}[3]{
		\addtocounter{footnote}{-#1}
		\footnotetext{#3\label{#2}}
		\addtocounter{footnote}{#1}
	}

	%esempio:
	% \mathscr{B}=\{\underset{\footnotemarkk{e1}}{\underbrace{v_1, \cdots, v_{k_1}}}, \underset{\footnotemarkk{e2}}{\underbrace{v_{k_1+1} , \cdots, v_{k_2+k_1}}}, \cdots, \underset{\footnotemark}{\underbrace{v_{k_{l-1}+kl }}}\}
	% \]
	% \consecfoottext{2}{e1}{autovettori rispetto a $ \lambda_1 $}
	% \consecfoottext{1}{e2}{autovettori rispetto a $ \lambda_2 $}
	% \footnotetext{autovettore rispetto a $ \lambda_{l}  $}
	% \todo{Manca matrice}

\usepackage{parskip} % rimuove l'indentazione dei nuovi paragrafi

%definizioni particolari
\newcommand{\straniero}[1]{\textit{#1}} %parole straniere
\newcommand{\titolo}[1]{\textsc{#1}} %titoli
%\newcommand{\qedd}{\makeatletter\displaymath@qed}
%QED
\newcommand{\qedd}{\tag*{$\square$}
	%\makeatletter\displaymath@qed\makeatother%
}

%citazioni
\usepackage{lineno}

\newcommand{\citazione}[1]{%
  \begin{quotation}
  \begin{linenumbers}
  \modulolinenumbers[5]
  \begingroup
  \setlength{\parindent}{0cm}
  \noindent #1
  \endgroup
  \end{linenumbers}
  \end{quotation}\setcounter{linenumber}{1}
  }
%

%rimuovere header e footer dalle pagine vuote
\usepackage{ifthen}
\makeatletter
\def\cleardoublepage{\clearpage\if@twoside \ifodd\c@page\else
    \hbox{}
    \vspace*{\fill}
    \vspace{\fill}
    \thispagestyle{empty}
    \newpage
    \if@twocolumn\hbox{}\newpage\fi\fi\fi}
\makeatother

% pacchetti matematica
\usepackage[intlimits]{amsmath} 
\usepackage{amssymb}
\usepackage{amsthm}
\usepackage{yhmath}
\usepackage{dsfont}
\usepackage{mathrsfs}

\usepackage{pgfplots} % stampare le funzioni
	\pgfplotsset{/pgf/number format/use comma,compat=1.15}
	%\pgfplotsset{compat=1.15} %per geogebra
	
\usepackage{cancel} % semplificare

\usepackage{polynom} %divisione tra polinomi

\usepackage{forest} % grafi ad albero

\usepackage{booktabs} % tabelle

\usepackage{commath} %simboli e differenziali

% definizione comandi matematici

\DeclareMathOperator{\epi}{Epi}
\DeclareMathOperator{\graph}{graph}
\DeclareMathOperator{\arcsec}{arcsec}
\DeclareMathOperator{\arccot}{arccot}
\DeclareMathOperator{\arccsc}{arccsc}
\DeclareMathOperator{\spettro}{Spettro}
\DeclareMathOperator{\nulls}{nullspace}
\DeclareMathOperator{\dom}{dom}
\DeclareMathOperator{\End}{End}
\DeclareMathOperator{\gl}{GL}
\DeclareMathOperator{\Id}{Id}
\DeclareMathOperator{\id}{Id}
\DeclareMathOperator{\I}{I}
\DeclareMathOperator{\rank}{rank}
\DeclareMathOperator{\tr}{tr}
\DeclareMathOperator{\tc}{t.c.}
\newcommand{\R}{\mathds{R}}
\newcommand{\K}{\mathds{K}}
\newcommand{\Q}{\mathds{Q}}
\newcommand{\N}{\mathds{N}}
\newcommand{\C}{\mathds{C}}
\newcommand{\Z}{\mathds{Z}}
\newcommand{\rmn}{\R^{m,n}}

\newcommand{\qmatrice}[1]{\begin{pmatrix}
#1_{11} & \cdots & #1_{1n}\\
\vdots & \ddots & \vdots \\
#1_{m1} & \cdots & #1_{mn}
\end{pmatrix}}

\newcommand{\parentesi}[2]{%
\underset{#1}{\underbrace{#2}}%
}

% Comandi per la creazione del riquadro attorno alle equazioni
% \equazione{arg1} crea una equazione con riquadro colorato

\usepackage[leqno,fleqn,intlimits]{empheq}
\usepackage[most]{tcolorbox}
\usepackage{ifthen}

	\newif\ifmarg
	\margtrue
	\ifmarg
	\makeatletter
	\let\mytagform@=\tagform@
	\def\tagform@#1{\maketag@@@{\mbox{~}\hbox{\rlap{\hspace{0.5in}(\ignorespaces#1\unskip\@@italiccorr)}}}\kern1sp}
	\renewcommand{\eqref}[1]{{\mytagform@{\ref{#1}}}}
	\makeatother
	\fi
	
	\newcommand*\mygraybox[0]{%
		\tcbhighmath}
		
	\newcommand{\equazione}[1]{	\begin{empheq}[box=\mygraybox]{equation*}
			#1
		\end{empheq}}

% \usepackage{geometry}
% \geometry{papersize={17.78cm,25.4cm}, bottom=3cm,%
% heightrounded}

\usepackage{hyperref}
\hypersetup{%
	pdfauthor={Davide Peccioli},
	pdfsubject={Appunti UniTO},
	allcolors=black,
	citecolor=black,
	colorlinks=true, 
	bookmarksopen=true}

% comandi per appunti

\newcounter{esempi}[section]
\newcounter{teorema}
\newcounter{proposizione}
\newcounter{osservazione}[section]
\newcounter{lemma}
\newcounter{corollario}

\newcommand{\esempi}[2][]{\stepcounter{esempi}\paragraph{Esempi #1 (\thesection.\theesempi)\label{es:\thesection.\theesempi}} #2}
\newcommand{\esempio}[2][]{\stepcounter{esempi}\paragraph{Esempio #1 (\thesection.\theesempi)\label{\thesection.\theesempi}} #2}
\newcommand{\proprieta}[2][]{\paragraph{Proprietà #1} #2}
\newcommand{\definizione}[2][]{\paragraph{Definizione #1} #2}
\newcommand{\notazione}[2][]{\paragraph{Notazione #1} #2}
\newcommand{\osservazione}[2][]{\stepcounter{osservazione}\paragraph{Osservazione (\thesection.\theosservazione)\label{oss:\thesection.\theosservazione}#1} #2}
\newcommand{\esercizio}[3][]{\paragraph{Esercizio #1} #2 
\paragraph{Soluzione} #3}
\newcommand{\nota}[1]{#1}

\usepackage[fulladjust]{marginnote} %to use marginnote for date notes

\newcommand{\teorema}[3][]{
    \stepcounter{teorema}
		\newcounter{thm#2}\addtocounter{thm#2}{\theteorema}
		\paragraph{Teorema
			\Roman{teorema} \label{thm:#2} #1} 
		#3}
\newcommand{\dimostrazione}[2]{\paragraph{\textit{dim.} \hyperref[thm:#1]{(\Roman{thm#1})}} #2}
\newcommand{\teoref}[1]{\hyperref[thm:#1]{\Roman{thm#1}}}

\newcommand{\proposizione}[3][]{
    \stepcounter{proposizione}
		\newcounter{prp#2}\addtocounter{prp#2}{\theproposizione}
		\paragraph{Proposizione
			\textit{p.}\roman{proposizione} \label{prp:#2} #1} 
		#3}
\newcommand{\dimostrazioneprop}[2]{\paragraph{\textit{dim.} \hyperref[prp:#1]{(\textit{p.}\roman{prp#1})}} #2}

\newcommand{\lemma}[3][]{
    \stepcounter{lemma}
		\newcounter{lmm#2}\addtocounter{lmm#2}{\thelemma}
		\paragraph{Lemma
			\textit{l.}\roman{lemma} \label{lmm:#2} #1} 
		#3}
\newcommand{\dimostrazionelem}[2]{\paragraph{\textit{dim.} \hyperref[lmm:#1]{(\textit{l.}\roman{lmm#1})}} #2}

\newcommand{\corollario}[3][]{
    \stepcounter{corollario}
		\newcounter{crl#2}\addtocounter{crl#2}{\thecorollario}
		\paragraph{Corollario\label{crl:#2} #1} 
		#3}
\newcommand{\dimostrazionecrl}[2]{\paragraph{\hyperref[crl:#1]{\textit{dim.} }} #2}

%%%%

\newcommand{\numerato}[1]{\begin{enumerate}
#1
\end{enumerate}}

\newcommand{\elencop}[1]{\begin{itemize}
#1
\end{itemize}}

\newcommand{\attenzione}[1]{
	\paragraph{Attenzione} #1
}

\newcommand{\conseguenza}[1]{
	\paragraph{Conseguenza} #1
}

\newcommand{\referenze}[2]{
	\phantomsection{}#2\textsuperscript{\textcolor{blue}{\textbf{#1}}}
}

\def\restriction#1#2{\mathchoice
              {\setbox1\hbox{${\displaystyle #1}_{\scriptstyle #2}$}
              \restrictionaux{#1}{#2}}
              {\setbox1\hbox{${\textstyle #1}_{\scriptstyle #2}$}
              \restrictionaux{#1}{#2}}
              {\setbox1\hbox{${\scriptstyle #1}_{\scriptscriptstyle #2}$}
              \restrictionaux{#1}{#2}}
              {\setbox1\hbox{${\scriptscriptstyle #1}_{\scriptscriptstyle #2}$}
              \restrictionaux{#1}{#2}}}
\def\restrictionaux#1#2{{#1\,\smash{\vrule height .8\ht1 depth .85\dp1}}_{\,#2}} %restringere dominio F

%numerazione equazioni
\renewcommand{\theequation}{\thesection.\arabic{equation}}
\numberwithin{equation}{section}

%TOdOS
\usepackage{tcolorbox}
\newcommand{\todo}[1]{%
\begin{tcolorbox}[colframe=red, colback=white]
	\centering
	#1
\end{tcolorbox}
}